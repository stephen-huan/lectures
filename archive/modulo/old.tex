\documentclass[11pt, oneside]{article}
\usepackage{titling, hyperref, geometry, amsmath, amssymb, algorithm, graphicx, textcomp, subcaption}
\usepackage[noend]{algpseudocode}
\usepackage[cache=false]{minted}
\geometry{a4paper}

\hypersetup{
  colorlinks=true,
  urlcolor=cyan
}

\newcommand{\emphasis}[1]{\textcolor{blue}{\textbf{\textit{#1}}}}

\title{Modulo}
\author{Stephen Huan}

\begin{document}
\maketitle

\section{Definition}

A number \( a \) \emphasis{modulo} \( m \) is defined to be the remainder after calculating \( \frac{a}{m} \),
written in math as \( a \mod m \equiv c \) but symbolized in programming as \( a \% m == c \).
Modulo is defined for integer \( a \) and \( m \) (i.e. either can be negative) but not for \( m = 0 \).
It is also considered a constant time calculation.

\subsection{Properties}

\[ a + b \mod m \equiv (a \mod m) + (b \mod m) \mod m \]
\[ a - b \mod m \equiv (a \mod m) - (b \mod m) \mod m \]
\[ a \cdot b \mod m \equiv (a \mod m) \cdot (b \mod m) \mod m \]

\section{Applications}

Commonly, problems that ask you to compute all possibilities will involve very large numbers,
thus they will ask you to compute the answer modulo \( m = 1000000007 \).

Make sure to output
\begin{minted}{python}
print(x if x >= 0 else x + m)
\end{minted}
as they expect the answer to be positive.

Modulo is also commonly used to implement ``cyclic'' behavior, because the values repeat with a period
of \( m \), to test divisibility, as \( a \mod m \equiv 0 \) implies \( a \) is divisible by \( m \), and is used
in many number theoretic algorithms, for example the Euclidean algorithm to find the greatest common divisor of two integers.

\begin{minted}{python}
def gcd(a, b):
    while b != 0:
        a, b = b, a % b
    return a
\end{minted}

\subsection{Modulo Exponentiation}
It is possible to compute \( b^e \mod m \) in \( O(\log e) \). Decompose \( e \)
into a binary number, that is, write it in the form \( \sum^{n - 1}_{i = 0} a_i 2^i \)
where \( a_i \) is 1 or 0 depending on whether the corresponding bit at that index is on or off and \( n \)
is the bitwise length of \( e \).
Then, \( b^e = b^{\sum^{n - 1}_{i = 0} a_i 2^i} = \prod^{n - 1}_{i = 0} b^{a_i 2^i} = \prod^{n - 1}_{i = 0} (b^{2^i})^{a_i} \)
by exponent rules. In order to calculate \( b^{2^i} \), note that \( b^{2^i} = b^{2 \cdot 2^{i - 1}} = b^{2^{i - 1} + 2^{i - 1}} = (b^{2^{i - 1}})^2 \).
Start with the base case of \( i = 0 \), which is \( b \), and then just square the base repeatedly.

The final algorithm is then
\begin{minted}{python}
def mod_exp(b: int, e: int, m: int) -> int:
    """ Returns b^e % m """
    if m == 1: return 0
    rtn = 1
    b %= m
    while e > 0:
        # bit on in the binary representation of the exponent
        if e & 1 == 1:
            rtn = (rtn*b) % m
        e >>= 1
        b = (b*b) % m
    return rtn
\end{minted}

\subsection{Modular Multiplicative Inverse}
I showed ways to efficiently calculate modulo under addition, subtraction, multiplication, and exponentiation.
How do you do division quickly?
The answer is to find some number \( x \) such that \( \frac{a}{b} \mod m \equiv a \cdot x \mod m \).
This number \( x \) is then the \emphasis{modular multiplicative inverse} of \( b \).
In order to compute \( x \), use the following algorithm.
\begin{minted}{python}
def extended_gcd(a: int, b: int) -> int:
    s, sp = 0, 1
    t, tp = 1, 0
    r, rp = b, a

    while r != 0:
        q = rp//r
        rp, r = r, rp - q*r
        sp, s = s, sp - q*s
        tp, t = t, tp - q*t

    return sp
\end{minted}
In Python,
\begin{minted}{python}
(a/b) % m == (a*extended_gcd(b, m)) % m
\end{minted}

\section{Sample Problems}

\begin{enumerate}
  \item \href{https://codeforces.com/group/M4wsRWBHyZ/contest/238084/problem/E}{2019 ICT In-house ``Havish's Exponents''}: \\
  Direct application of modular exponentiation.

  \item \href{https://codeforces.com/group/M4wsRWBHyZ/contest/238084/problem/G}{2019 ICT In-house ``Permutations''}: \\
  Find the number of permutations for a given string with length \( n \).

  Solution: The number of permutations will be \( n! \) divided by \( \prod_{ch} c_{ch}! \) where
  \( c_{ch} \) is the number of times a particular character \( ch \) appears in the string.

  Calculate \( n! \) by doing repeated multiplication while modding to keep it small. Then, precompute
  the modular multiplicative inverse for each size \( i \) up to \( n \). Finally, for each distinct
  character of the original string, count how many times it appears in the string and multiply \( n \)
  by the multiplicative modular inverse of each number \( i \) up to the count. Initially, this seems to be a \( n^2 \) algorithm but note that the sum of the counts is exactly \( n \), thus the algorithm does \( O(n) \) work.

\end{enumerate}

\section{Works Cited}

\begin{enumerate}
  \item \href{https://en.wikipedia.org/wiki/Modular_exponentiation}{Wikipedia - ``Modular exponentiation''}
  \item \href{https://en.wikipedia.org/wiki/Extended_Euclidean_algorithm}{Wikipedia - ``Extended Euclidean algorithm''}
\end{enumerate}

\end{document}
