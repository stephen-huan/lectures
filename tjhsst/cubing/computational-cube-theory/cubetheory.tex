\documentclass[11pt, oneside]{article}
\usepackage{indentfirst, hyperref, geometry, amsmath, amssymb, algorithm}
\usepackage[noend]{algpseudocode}
\usepackage[cache=false]{minted}
\geometry{a4paper}

\hypersetup{
  colorlinks=true,
  urlcolor=cyan
}

\title{Various Computational Cube Theory Conjectures}
\author{Stephen Huan}
\date{July 6, 2019}

\begin{document}
\maketitle

\section{Basics}

\subsection{What This Is}
This lecture will not directly make you faster. Nor is it really about
Rubik's cubes. Instead, it is about general computer algorithms that
currently exist and I postulate will exist. If you appreciate listening
to someone talk about cubes, I encourage you to write your own lectures
and make Rubik's Cube Club great again. In the words of Jayden McNeill,
``Don't be a cubing intellectual''.

\subsection{Representation}
At first, I imagined doing a sticker-wise face representation. That is, an
array of shape 6x3x3. The first dimension represents the face (there are 6
faces on a cube). Note that I specify a face by color in WCA orientation (white
top green front) and colors by their first letter (they are distinct); the
ordering is WGRBOY: a counterclockwise-esque rotation pattern. Then, for each
face, there are 9 stickers. I then subdivided these 9 stickers into 3 rows of
3. The problem is that simulating a turn becomes too unwieldy. A single move
will shift 12 stickers along its axis, which are hard to keep track of, and
shifts 9 stickers which are easier to cycle. In total 21 stickers move, and
it's hard to generalize for the 6 possible moves.

An obvious alternative I realized when reading
\href{https://github.com/hkociemba/RubiksCube-TwophaseSolver}{Kociemba's
GitHub} is a cubie-level representation, which simplifies moving. In my
implementation a cubie contains 3 ordered colors. Cubie(W, B, O) means the
white is either up/down (U/D), the blue is front/back (F/B), and the orange
is left/right (L/R). In general, I also refer to moves by the equivalent
face they are done to. The ordering is a representation of the concept
of ``orientation''. The logic is the same for edge pieces. For example,
Cubie(W, B, None) is a valid edge. The white center would be Cubie(W, None,
None) and the even core of the cube would be Cubie(None, None, None).

As you can see the same class can represent each piece type. Now let's simulate
a move. First, track a corner; in this case the WBO corner. When you turn the
U layer the W remain facing U/D, but the O faces F/B and the B faces L/R. Now
track the WGR corner as I do a R move. R remains facing L/R, but W now faces
F/B and G faces U/D. Suppose U/D is 0, F/B is 1, and L/R is 2. A move will not
affect its corresponding color, and will swap the two remaining colors.

\begin{minted}{python}
def rotate(self, dir):
    i, j = {0, 1, 2} - {dir}
    self.colors[i], self.colors[j] = self.colors[j], self.colors[i]
\end{minted}

Now that I know how moves affect cubies, I just need to move the cubie objects
themselves. First of all, I will make an array to hold cubie objects, this
time in layers. I started with the top layer, going from left to right, top
to bottom. This array has shape 6x9, as I felt it was no longer necessary
to subdivide further into rows. To do a move, get the layer it corresponds
to. Then, cycle the corners and cycle the edges. Finally, apply the rotation
algorithm shown above on each piece in the layer. Note that when I ``get''
a layer I return indices, not the objects themselves. I need to modify the
array, and since Python doesn't have pointers, only indices will do. Getting
these indices is remarkably easy.

Here is my cycle algorithm. Note that it is extremely
general, as I might change the dimensionality in the future.
\begin{minted}{python}
def cycle(l, indexes):
    start = access(l, indexes[-1])
    for i in range(len(indexes) - 1, 0, -1):
        modify(l, indexes[i], access(l, indexes[i - 1]))
    modify(l, indexes[0], start)
\end{minted}

Finally, here is how I perform a move. Note that CWC
and CWE are lists of indices specifying the order.
\begin{minted}{python}
def move(self, dir):
      layer = self.get_layer(dir)

      cycle(self.cube, get(layer, CWC))
      cycle(self.cube, get(layer, CWE))

      for cubie in layer: access(self.cube, cubie).rotate(ORIENT[dir])
\end{minted}

Some details: I specify color using numeric codes, because:
\begin{enumerate}
  \item The color ordering is now WYGBRO.
  \item The opposite color to a given color \( c \) is \( 5 - c \), like dice.
  \item The orientation of a move is \( j//2 \), where \( j \) is the index of the move in the ordering.
\end{enumerate}

Of course, you could use strings or whatever
to your liking. It doesn't really matter.

\subsection{Commentary}
Evidently, a cubie level approach is superior to a
sticker level approach. I have some intuition as to why.
\begin{itemize}
  \item Traditional speedsolving methods emphasize blockbuilding and a
    piece level approach. We like to make fun of beginners for saying ``I
    can only solve a face'', as it shows a fundamental näivety with the
    geometry of the puzzle.
  \item In general, the more abstract something
    is, the easier it is to manipulate.
\end{itemize}

If you look at the structure of my code, available
\href{https://github.com/stephen-huan/Cube-Solver}{here} you'll realize I like
named constants, small, modular, functions, and headers. I call this ``AI
style''. I do not write production code. As such, I think this format is good
enough for a beginner playing around at programming. As a side note, you, the
reader, should take AI. It is by far the greatest class TJ has offered me in
these two long and stressful years and has provided a foundation that will last
me the rest of my career. This lecture would not be possible otherwise.

\section{High Level Algorithmic Overview}

\subsection{Problem Structure}
Given a cube state, the problem is to find the shortest move sequence such
that the cube is solved after the sequence. That is, I want to solve a cube in
the fewest moves possible. This problem is a graph problem. The vertices are
states of the cubes, and the edges are possible moves. Obviously this graph
has cycles, but we can prune those out. Now, this is a single-source-shortest
path problem, on an undirected graph.

\subsection{Uninformed Searches}

\subsubsection{BFS}
Imagine copying the cube 18 times and then applying U to one of the copies, R
to another, etc. until I have 18 distinct cubes. Then for each of those cubes,
I do the same thing. If I enumerate every single possibility, I will of course
find the shortest solution. This algorithm is called breadth-first-search or
``BFS'' and it has the computational complexity of \( O(|V| + |E|)\) as in the
worse case I visit each vertex and each edge exactly once. In this case, since,
the graph is a tree, \( |E| = |V| - 1\). Therefore the complexity is \( O(|V|
+ |V| - 1)\) = \( O(|V|) \).

That seems pretty good, as it's linear in \( |V| \). Unfortunately, the
number of vertices grows at an exponential rate. It's more accurate to say
the complexity is \( O(b^d) \), where \( b \) is the branching factor, the
number of moves at each depth, and \( d \) is the depth.

If I'm in half turn metric (HTM) then there are 15 (6*3, but minus 3 since it
doesn't make sense to do a move and then a move on the same face) possible
moves at each step. God's number is 20, so in the worst case I'd have to
search \( 15^{20} \) states (approximately \(3.3 \cdot 10^{25} \) states)
which is way too large. Clearly, I need a faster algorithm.

\subsubsection{Bi-BFS}
A simple, but enormously effective optimization. Do a BFS, but from start
to goal and goal to start simultaneously and symmetrically. When the two
searches meet in the middle, just combine the two paths. Intuitively, two
small trees will halt the exponential growth in half. When I solve mazes,
I'll jump from end to end and try to combine the two. The new complexity
is \( O(b^{d/2}) = O(\sqrt{b^d}) \), doubling my effective search depth.

Some subtleties in implementation: adding children to seen is much
faster, but don't goal test on children. My implementation is extremely
slow though: it takes 30 seconds to find the optimal 9 HTM U-perm.

\subsubsection{IDDFS}
Essentially the same thing as BFS but uses less memory. Since DFS goes
straight down, only the path from start to current node needs to be stored.
Therefore, the memory consumption is \( O(d) \). The trade off is time, as
iterative deepening means running DFS to depth 0, then 1, then 2, etc. until
depth \( d \). The complexity is \( O(1 + b^1 + b^2 + b^3 + ... + b^d) \).
Using geometric series this sums to \( \frac{b^{d + 1} - 1}{b - 1} = \lim_{b
\to \infty} b^d\), because of the definition of big-O (asymptotic).

An added benefit is it avoids the copying cost, which is a nontrivial
factor (\#1 in total time when profiled). Instead of copying a cube
object to generate children, just stick with one cube object and
perform a single move when going down, and doing a single inverse
move when going back up. With a BFS, I would need to apply multiple
inverse moves, making it slightly slower than just copying.

A comparison of all the algorithms discussed so far:

\begin{tabular} {| c | c | c | c | c |}
  \multicolumn{5}{c}{Algorithms} \\ \cline{2-5}
  \multicolumn{1}{c |}{} & BFS & Bi-BFS & IDDFS & Bi-IDDFS \\ \hline
  Time & \( O(b^d) \) & \( O(b^{d/2}) \) & \( O(b^d) \) & \( O(b^{d/2}) \) \\ \hline
  Space & \(O(b^d) \) & \(O(b^{d/2}) \) & \( O(d) \) & \( O(d) \) \\ \hline
\end{tabular}

\subsection{Informed Searches}
We can do better. If I take advantage of a prediction of how far away a given
state is to the goal, then the search is faster. The general algorithm for this
is called A*, and even though it uses a heuristic, the search will return an
optimal path if the heuristic is admissible. Optimality proof TODO.

\section{What Exists}
The undisputed best algorithm for solving 3x3 is, of course,
\href{http://kociemba.org/cube.htm}{Kociemba's algorithm}. It
works by some horrendously complicated two-phase group reduction
math stuff. A ``group'', as I understand it, is a subset of cube
states able to be solved by a subset of moves.

\newpage

Drawbacks:
\begin{itemize}
  \item Not generalizable: 4x4+ group reduction would be intractable (necessary to do slice moves)
  \item Fundamentally a search algorithm: Stockfish was made a decade ago.
  \item Takes \( \approx 12 \) seconds to solve a cube optimally.
\end{itemize}

\section{The Future}
Now, there exists machine learning. ML traditionally deals with vectors, so
I propose one-hot encoding colors and then flattening into a linear order,
sticker by sticker face by face. The resulting dimensionality would be \(
6 \cdot 9 \cdot 6 = \) 324x1. The reason I don't use numerical color codes
is because that causes numerically similar colors to be literally closer to
each other, which is something I don't want. Therefore, I make the colors
orthogonal to each other.

\subsection{Pure Prediction}
The most direct approach would be to have a neural network (NN) or
convolutional neural network (CNN) directly predict the next move given a
cube state. The problem would be probable difficulty in extracting a viable
feature space and I have no idea how good this would actually be. Also, there'd
be no way to verify optimality on a input that hadn't been solved before.

\subsection{Boosting A*}
I mentioned earlier that A* required an admissible heuristic. The determination
of such a function is nontrivial and NNs could simulate a heuristic. The
problem would be the lack of guarantee of admissibility, which is solved by
the following approach. First off, add a regularization factor in loss to
penalize overestimation more than underestimating. This by itself does not
guarantee admissibility, thought it encourages it. Suppose the heuristic
is \( y' \) and the actual distance is \( y \). I want a function \( f \)
such that \( \forall y' f(y') \leq y' \) and \( f(y') < y \). That is, if a
function satisfies both conditions, then it must be admissible. A ``trivial''
function that accomplishes this is \( f(y') = y' - e \), where \( e \) is
\( \max_{\forall y'} y' - y \). However, this linear ``force'' layer may be
too aggressive, as I want the heuristic as large as possible. Stack NNs on
top of each other successively to soften the blow, then add the force layer
to force admissibility.

\subsection{Reinforcement Learning}
In my opinion, this has the greatest viability. AlphaGO is good example
of RL's power. The combination of a small discrete action space coupled
with a massive state space is very comparable to Chess, Othello, Go, etc.

The added benefit is the Rubik's cube is arguably much easier than Chess
or Go, which have probably 5x the action space and depths in the hundreds
to thousands. Such a system would be extremely general as applying it to
4x4 is as easy as generating more training data.

\end{document}
