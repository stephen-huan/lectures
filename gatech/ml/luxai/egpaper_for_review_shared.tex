% SIAM Shared Information Template
% This is information that is shared between the main document and any
% supplement. If no supplement is required, then this information can
% be included directly in the main document.


% Packages and macros go here
\usepackage{lipsum}
\usepackage{amssymb}
\usepackage{amsfonts}
\usepackage{amsopn}
\usepackage{mathtools}
\usepackage{graphicx}
\usepackage{epstopdf}
\usepackage{algorithmic}
\usepackage{tikz}
% \usepackage{pgfplots}
\usetikzlibrary{shapes.arrows, patterns, calc}
\usepackage{tikz-3dplot}
\usepackage{bbm}
\usepackage{bm}
\ifpdf
  \DeclareGraphicsExtensions{.eps,.pdf,.png,.jpg}
\else
  \DeclareGraphicsExtensions{.eps}
\fi
% \usepackage{todonotes}

\definecolor{lightblue}{HTML}{a1b4c7}
\definecolor{orange}{HTML}{ea8810}
\definecolor{silver}{HTML}{b0aba8}
\definecolor{rust}{HTML}{b8420f}
\definecolor{seagreen}{HTML}{23553c}

\colorlet{darklightblue}{lightblue!85!black}
\colorlet{darkorange}{orange!85!black}
\colorlet{lightsilver}{silver!15!white}
\colorlet{darksilver}{silver!85!black}
\colorlet{darkrust}{rust!85!black}
\colorlet{darkseagreen}{seagreen!85!black}

\definecolor{ice}{HTML}{48dbfb}
\definecolor{ore}{HTML}{2c3e50}
\definecolor{rubblelight}{HTML}{f4f4f4}
\definecolor{rubbledark}{HTML}{9f9f9f}

\definecolor{factory1}{HTML}{5aa8ec}
\definecolor{lichen1}{HTML}{90c2ed}
\definecolor{robot1}{HTML}{228be6}

\definecolor{factory2}{HTML}{f46f6f}
\definecolor{lichen2}{HTML}{f2a8a8}
\definecolor{robot2}{HTML}{f03e3e}

% Names of standard objects
\newcommand*{\Reals}{\mathbb{R}}
\newcommand*{\Naturals}{\mathbb{N}}

\newcommand*{\defeq}{\coloneqq}
\newcommand*{\BigO}{\mathcal{O}}
\newcommand*{\N}{\mathcal{N}}
\newcommand*{\SpSet}{\mathcal{S}}
\newcommand*{\GP}{\mathcal{GP}}
\newcommand*{\Loss}{\mathcal{L}}
\newcommand*{\Order}{\mathcal{I}}
\newcommand*{\Reverse}{\updownarrow}
\newcommand*{\I}{I}
\newcommand*{\J}{J}
\newcommand*{\V}{V}

\renewcommand*{\vec}[1]{\bm{#1}}
\newcommand*{\Id}{\text{Id}}

% Names of variables
% covariance matrix
\newcommand*{\CM}{\Theta}
\newcommand*{\mean}{\mu}
\newcommand*{\var}{\sigma^2}
\newcommand*{\std}{\sigma}
% kernel function
\newcommand*{\K}{K}
\newcommand*{\Train}{\text{Tr}}
\newcommand*{\Pred}{\text{Pr}}

% Names of operators
\DeclarePairedDelimiter{\norm}{\lVert}{\rVert}
\DeclarePairedDelimiter{\card}{\lvert}{\rvert}
\DeclarePairedDelimiter{\ceil}{\lceil}{\rceil}
\DeclareMathOperator{\diag}{diag}
\let\trace\relax
\DeclareMathOperator{\trace}{trace}
\DeclareMathOperator{\logdet}{logdet}
\DeclareMathOperator{\chol}{chol}
\DeclareMathOperator{\FRO}{FRO}

\DeclareMathOperator*{\argmin}{argmin}
\DeclareMathOperator*{\argmax}{argmax}

\DeclarePairedDelimiterX{\infdivx}[2]{(}{)}{%
  #1\;\delimsize\|\;#2%
}
\newcommand*{\KL}{\mathbb{D}_{\operatorname{KL}}\infdivx}
\DeclareMathOperator{\p}{\pi}
\DeclareMathOperator{\E}{\mathbb{E}}
\DeclareMathOperator{\Var}{\mathbb{V}ar}
\DeclareMathOperator{\Cov}{\mathbb{C}ov}
\DeclareMathOperator{\Corr}{\mathbb{C}orr}
\DeclareMathOperator{\entropy}{\mathbb{H}}
\DeclareMathOperator{\MI}{\mathbb{I}}
\DeclareMathOperator{\Prob}{\mathbb{P}}
\DeclareMathOperator{\Ind}{\mathbbm{1}}
\DeclareMathOperator{\cw}{\mathsf{cw}}
\DeclareMathOperator{\Uniform}{\text{Unif}}

%%% Local Variables:
%%% mode:latex
%%% TeX-master: "experimental_design_linalg"
%%% End:
